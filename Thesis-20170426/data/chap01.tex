\chapter{绪论}
\label{cha:introduction}
\section{论文研究背景及意义}
随着信息技术的不断发展,21世纪我们已经进入了一个“信息爆炸”的社会,而且在近几年我们的社会几乎进入互联网时代后,社会中的每个人每天接收到的咨询和信息以及各种各样的互联网产品同互联网发展的初期相比,超出了太多太多。~\cite{林薇2015基于}。另外,我国的互联网技术在近十几年内不断的发展,开发WEB产品的人力成本和技术成本不断降低,用户对于 WEB 应用的需求\cite{邵志强2015跨媒体检索平台}不再满足于通过缓慢的加载和复杂的操作获取资讯~\cite{贺海梁2012基于},
许多的创业者希望通过开发一个WEB应用实现自己创业梦想的时代已经无法获得投资人的青睐。而且,由于近几年WEB信息泄露的新闻不断进入消费者的视野,用户对于WEB应用的安全性体验也提出了更高的要求。

随着互联网技术进一步普及,基于模型-视图-控制器
(Model-View-Controller, MVC)模式的WEB应用程序~\cite{张2010基于}被广大开发者所使用,目前主流的WEB应用程序均在使用此框架。使用MVC框架,可以将WEB应用进行清晰的分层开发,前端开发人员主要负责页面的呈现方式和用户体验,后台人员则主要负责应用的逻辑实现以及数据结构的搭建,极大的降低了研发成本。

随着信息化的进一步普及,数据库的使用越来越广泛,数据成为一个应用甚至也个企业最重要的价值体现,越来越多的企业发展离不开数据库。因此数据库的稳定性和安全性也称为很多企业研发的重点。通过设计数据的管理和使用机制,在提高数据库使用效率的同时,保障数据的完整性和安全性是企业运维人员的在运维过程中的重要任务。

除此之外,如何实现WEB应用的高可用性和分布式服务也是保障用户体验的一个很重要的部分,通过设计基于应用和服务器的不同层级不同纬度的优化策略,提升应用的高可用性也是在WEB应用开发中必须要注意的。

\section{国内外研究现状}
从上世纪九十年代开始,计算机技术和Internet互联网技术开始迅速发展, 随着这些技术的不断发展,在互联网络中网络信息的存储量级和访问数量都是以几何数级进行增长,随之而来导致的问题就是网络访问的拥塞问题和网络服务的超载运行~\cite{王霜2004web}。

2006年新华网被黑事件、2010年的百度域名劫持时间、2011年的CSDN用户数据泄漏事件以及2015年网易邮箱密码泄漏事件等事件无不说明系统优化的作用和意义。

目前大部分互联网产品在开始发展的阶段都重视了产品的设计和功能,然而在产品的稳定性和安全性方面却考虑不足,这在一定程度上增加了互联网产品在推广过程中潜在的风险。因此,在开发互联网产品的同时,通过一系列的优化策略,提升应用的体验,增强应用的稳定性和安全性必须得到充分的认识。

\section{论文主要研究内容}
\subsection{论文主要工作内容}
本篇论文通过在满足本人参与的WEB应用正常运行的基础上,通过应用、数据和服务等不同维度的的优化实践研究基于Spring MVC架构WEB应用的系统优化策略。首先通过搭建基于Spring MVC的WEB测试应用,在实际用户使用的过程中通过不断开发和调整系统的优化策略,提升应用的用户体验和应用的稳定性,探索行之有效的系统优化策略。

本文的研究对象主要有以下内容:

(1) 对Jenkins持续集成环境研究。自动化部署和代码检测是保证基于WEB的应用产品质量的一个重要环节,通过研究部署Jenkins集成测试环境,对编写的代码进行版本控制、自动化构建和代码测试,研究持续集成方案对于WEB应用系统优化策略的影响。

(2) 对Couchbase缓存机制进行研究。目前大多数的WEB应用对于缓存性能的优化还没有足够的重视,通过开发针对WEB应用的Couchbase缓存系统,研究Couchbase的缓存机制测试有效的缓存机制对于系统性能提升和用户体验的影响,探索可用的系统优化策略。

(3) 对Docker容器编排技术进行研究。随着用户的不断增加,单一节点的WEB应用或数据库应用已经无法满足用户的需求,如何快速的部署新的应用节点提升应用性能成为开发者关注的问题。通过构建基于Docker容器编排技术的应用容器,在新的服务器快速部署新的应用,并加入到应用集群中去,探索Docker机制对于系统优化策略的影响。

(4) 对数据库的主从复制进行研究。随着应用的发展,数据逐渐成为应用最有价值的部分,如何更好的保证数据的完整性以及安全性在应用的维护过程中日益重要。通过研究数据库的主从复制和延时复制方案,探索数据库优化对于系统安全性优化的策略。

(5) 对基于API的监控和应急措施方案进行研究。随着服务节点的不断增加,服务的高可用和服务的健康性检查时WEB应用的运维人员在维护过程中必须要注意的方面。通过开发基于API的服务器、服务监控系统以及基于API的应急处理系统研究API操作对于系统的快速检测和快速恢复的影响。

(6) 其他方面,研究WEB应用的搭建过程、Tomcat的相关配置、负载均衡以及阿里云的相关配置研究WEB应用和服务器自身优化对于系统性能的影响。

\subsection{论文目标}
本论文致力于分析目前很多WEB应用开发过程中存在的问题和不完善的地方,并探索WEB应用性能和服务器性能的优化方案。

在Linux平台上,使用JAVA语言和MySQL数据搭建一个基于Spring MVC架构的WEB应用,通过Tomcat实现应用的访问和测试,通过设计不同的优化方案对WEB进行测试,研究在实际应用中有价值的优化策略。

系统性能优化策略研究主要包含应用层面、数据层面和服务器层面三个层面。在应用层面,通过持续集成、代码审核实现应用的稳定性和安全性。在数据库层面,通过多数据库主从复制、负载均衡和备份恢复等策略实现数据的高可用和稳定性。在服务器层面,通过对服务的监控和服务器的监控、服务器间负载均衡的配置、基于API的自动化failover等策略保障服务的正常运行和高负载应对。
\section{论文组织结构}
基于Spring MVC架构WEB应用的系统优化策略的研究是按照计划逐步完成的,本人将此研究分为六个部分:

第一章:绪论。本章主要介绍WEB应用开发的现状和系统优化策略研究的意义,明确了在WEB应用开发过程中对于应用和服务器进行优化的必要性和重要性。除此之外,本章还介绍了论文主研究和实验的主要内容、论文的目标、论文创新点以及本篇论文的主要结构。

第二章:WEB应用开发介绍。本章主要介绍了前端和后端的开发框架和开发流程以及数据库的搭建过程。然后介绍了Tomcat的配置过程。之后介绍了Jenkins持续集成环境的搭建和使用过程。

第三章:应用性能优化介绍。本章主要介绍了在WEB应用性能优化方面册主要方法和策略,主要包括Couchbase缓存优化,Tomcat高性能Apr配置,Docker分布式优化以及SLB负载均衡优化等方面。同时对应用的安全性和稳定性进行了分析。

第四章:数据优化。本章主要介绍了MySQL数据库在设计和开发过程的优化策略,主要包括基于数据稳定性的复制方案,基于高可用的负载均衡方案,基于数据完整性的备份和恢复方案。同时总结了出现问题时的解决方案。

第五章:服务监控与应急措施优化。本章主要介绍了服务器层面的优化策略,主要包括多服务器的心跳监听服务配置,服务性能的监控脚本实现,基于API的应急措施处置以及服务器安全性配置等方面的优化策略。同时总结了多种方案整合使用的策略。

第六章:总结与展望。本章主要总结了不同优化策略对于系统性能的影响以及在研究过程中获得的经验,并对优化策略的进一步研究做了简单的展望。