\chapter{绪论}
\label{cha:introduction}
\section{论文研究背景及意义}
随着信息技术的不断发展,21世纪我们已经进入了一个“信息爆炸”的社会,而且在近几年伴随着我们进入互联网时代后,每个人每天接收到的咨询和信息以及各种各样的互联网产品同互联网发展的初期相比截然不同~\cite{林薇2015基于}。另外,我国的互联网技术在近十几年内不断的发展,开发WEB产品的人力成本和技术成本不断降低,用户对于 WEB 应用的需求\cite{邵志强2015跨媒体检索平台}不再满足于通过缓慢的加载和复杂的操作获取资讯~\cite{贺海梁2012基于},
以至于现如今许多的创业者仅通过开发一个WEB应用来实现创业的目的是远远不够的,更不能引起投资人的兴趣和注意。而且,由于近几年WEB信息泄露的事件层出不穷,用户对于WEB应用的安全性体验也提出了更高的要求。

随着互联网技术进一步普及,基于模型-视图-控制器
(Model-View-Controller, MVC)模式的WEB应用程序~\cite{张2010基于}被广大开发者所使用,目前主流的WEB应用程序均在使用此框架。使用MVC框架,可以将WEB应用进行清晰的分层开发,前端开发人员主要负责页面的呈现方式和用户体验,后台人员则主要负责应用的逻辑实现以及数据结构的搭建,极大的降低了研发成本。

随着信息化的进一步普及,数据库的使用越来越广泛,数据成为一个应用甚至一个企业最重要的价值体现,越来越多的企业发展离不开数据库。因此数据库的稳定性和安全性也成为很多企业研发的重点。通过设计数据的管理和使用机制,在提高数据库使用效率的同时,保障数据的完整性和安全性是企业运维人员在运维过程中的重要任务。

除此之外,如何实现WEB应用的高可用性和分布式服务也是保障用户体验的一个很重要的部分,通过设计基于应用和服务器的不同层级不同维度的优化策略,提升应用的高可用性也是在WEB应用开发中必须要注意的。

\section{国内外研究现状}
从上世纪九十年代开始,计算机技术和Internet互联网技术开始迅速发展, 随着这些技术的不断发展,在互联网络中网络信息的存储量级和访问数量都是以几何数级进行增长,随之而来导致的问题就是网络访问的拥塞问题和网络服务的超载运行~\cite{王霜2004web}。

在应用的开发方面,目前国内的项目开发团队采取的一般策略是先进行应用的设计和开发,然后投入到市场中进行快速的验证,希望通过短时间的快速迭代来对应用和服务进行优化,这种方式固然有非常大的优势,可以对产品的功能和设计进行快速的验证并且降低项目的成本。但是,在实际的市场环境中,以上方法也会造成非常严重的后果,比如因为数据库的安全性配置导致的数据丢失和泄漏事件,因为应用的自身配置导致的高延迟低负载现象,因为服务器的配置、优化和监控方面没有设计导致的服务中断等现象,这些问题都增加了互联网产品在推广过程中的潜在风险,因此在开发互联网产品的过程中,通过一系列的优化策略,提升应用的体验,增强应用的稳定性和安全性必须得到充分的重视。

\begin{enumerate}
  \item 持续集成

  持续集成(CI)的目标是对开发团队的代码进行集成,包括代码的构建、单元测试与集成测试的执行,以及生成执行结果的报表。目前主流的持续集成工具包括传统的CruiseControl和Jenkins以及云计算环境中的Travis CI等,其中Travis CI要求软件的所有源代码在Github上托管后进行持续集成,这对于许多商业环境中的保密项目来说有很大的局限性。CruiseControl和Jenkins都是基于Java开发的开源软件,2001年推出的CruiseControl可以实现代码的集成、构建和单元测试等基本功能,也可以实现邮件通知、Ant以及对各种源代码控制系统的支持,但是在2004年推出的Jenkins在三到四年的市场验证中超过了CruiseControl,逐渐成为了最流行的CI工具,它除了包含CruiseControl所具有的基本功能以外,还支持集成测试和插件开发,通过使用官方或者自定义的插件,可以为代码的集成部署提供更为自由的支持,使用者可以通过插件实现多语言项目的统一构建、代码的远程部署以及容器化的部署等功能,大大提高了应用的稳定性和开发者的部署效率。
  \item 应用性能优化

  在应用的开发过程中,对于应用的缓存机制开发和高并发支持是应用优化的两个主要方面。

  通过缓存技术,对系统的数据进行预加载,提升了应用的访问效率。目前业界使用得最多的 Cache 系统主要是 memcached 、redis和Couchabse,其中memcached在集群和持久化方面支持的不够,redis虽然支持持久化,但是会造成系统间歇性的负载很高的现象,因此在实际的生产环境中会避免使用上述系统来作为系统的缓存。Couchabse作为近几年比较流行的缓存系统,它吸取了memcached的众多优点,并且在集群的支持、负载均衡和高可用方面都进行了开发和优化,在实际使用过程中能够更加灵活的配置和扩容,而且有容器化的镜像,这对于DevOps来说能够提升运维开发人员的工作效率和系统的容错性。

  通过对于应用的高并发支持,可以使应用在面对大并发的情况下有效保证系统的负载。目前最常用的Servlet容器有Jetty和Tomcat,二者在性能方面差别不大,Jetty相对来说比较轻量且架构比较简单,适合快速开发和部署,但是Tomcat对JavaEE和Servlet的支持更加全面,很多特性会直接集成进去。因此在处理复杂业务和请求时优势明显,而且Tomcat在处理高并发的请求时,有成熟的应对方法,通过开启APR模式可以有效的应对应用的高并发情况。
  \item 数据优化

  数据对于应用来说是最重要的价值体现,目前在数据库优化方面主要有两种方法,一种是第三方的优化方法,一种是基于自身数据库引擎的优化。采用第三方的优化方法有许多功能和扩展方式,可以从多个角度进行数据库的优化和复制,但是第三方工具的缺点是稳定性不足,当工具崩溃后会导致数据的丢失或延迟;基于自身的优化虽然没有个性化的功能,但是稳定性较高,而且随着版本的不断升级,数据库自身的功能也在不断的丰富和完善,对于MySQL而言,最新的版本在数据处理效率和数据复制方面的表现均优于第三方工具,但是需要不断的去调整参数使性能达到最优。
  \item 服务健康检测与恢复

  除了应用和数据之外,系统和服务的稳定性对于互联网产品而言也是非常重要的,因此需要探索服务的健康检测和故障恢复方案。目前我们可以通过专业的第三方平台如阿里云的监控服务进行,可以通过使用监控程序如zabbix来实现,还可以通过自己开发监控脚本来实现。对于专业的监控工具,他们有专业的组件可以实现用户对于服务的监控需求,但是对一些具体的应用接口和自定义服务,这些工具的支持不足,另外专业监控工具需要单独的安装和配置,这些相对于自己开发的脚本而言,缺乏灵活性。目前,随着脚本语言Python的发展和第三方工具的支持,通过设计开发自定义的监控脚本不但可以降低软件的安装配置成本,而且可以增加灵活性,对于故障的恢复也可以通过脚本进行远程实现,因此本论文将自己开发监控和故障恢复脚本作为服务健康检测和故障恢复的首选。对于第三方的监控平台,考虑到配置简单和成本低等方面,可以跟自己开发的监控脚本相配合,实现最完整的服务健康检测和故障恢复。
\end{enumerate}

目前在系统优化策略研究方面,很多的企业做了大量的工作,但是有很多的优化策略都是针对于单一方面进行的,从多个方面进行协同优化研究的文章和案例并不多,这对于我们进行系统化的优化研究缺少相应的参照和分析。

\section{论文主要研究内容}
\subsection{论文主要工作内容}
本篇论文在保证本人参与的WEB应用正常运行的基础上,通过应用、数据和服务等不同维度的的优化实践研究基于Spring MVC架构WEB应用的系统优化策略。首先通过搭建基于Spring MVC的WEB测试应用,在实际用户使用的过程中通过不断开发和调整系统的优化策略,提升应用的用户体验和稳定性,探索行之有效的系统优化策略。

本文的研究对象主要有以下内容:

(1) 对Jenkins持续集成环境进行研究。自动化部署和代码检测是保证基于WEB的应用产品质量的重要环节,通过研究部署Jenkins集成测试环境,对编写的代码进行版本控制、自动化构建和代码测试,研究持续集成方案对于WEB应用系统优化策略的影响。

(2) 对Couchbase缓存机制进行研究。目前大多数的WEB应用对于缓存性能的优化有所欠缺,通过开发针对WEB应用的Couchbase缓存系统,研究Couchbase缓存机制对于系统性能提升和用户体验的影响,探索可用的系统优化策略。

(3) 对Docker容器编排技术进行研究。随着用户的不断增加,单一节点的WEB应用或数据库应用已经无法满足用户的需求,如何快速的部署新的应用节点提升应用性能成为开发者关注的问题。通过构建基于Docker容器编排技术的应用容器,在新的服务器快速部署新的应用,并加入到应用集群中去,探索Docker机制对于系统优化策略的影响。

(4) 对数据库的主从复制进行研究。随着应用的发展,数据逐渐成为应用最有价值的部分,如何更好的保证数据的完整性以及安全性在应用的维护过程中日益重要。通过研究数据库的主从复制和延时复制方案,探索数据库优化对于系统安全性优化的策略。

(5) 对基于API的监控和应急措施方案进行研究。随着服务节点的不断增加,服务的高可用和服务的健康性检查是WEB应用的运维人员在维护过程中必须要注意的方面。通过开发基于API的服务器、服务监控系统以及基于API的故障恢复系统来研究API操作对于系统的快速检测和故障恢复的影响。

(6) 其他方面,研究WEB应用的搭建过程、Tomcat的相关配置、负载均衡以及阿里云的相关配置研究WEB应用和服务器自身优化对于系统性能的影响。

\subsection{论文目标}
本论文致力于分析目前很多WEB应用开发过程中存在的问题和不完善的地方,并探索WEB应用性能和服务器性能的优化方案。

在Linux平台上,使用JAVA语言和MySQL数据搭建一个基于Spring MVC架构的WEB应用,通过Tomcat实现应用的访问和测试,通过设计不同的优化方案对WEB进行测试,研究在实际应用中有价值的优化策略。

系统性能优化策略研究主要包含应用层面、数据层面和服务器层面。在应用层面,通过持续集成、代码审核实现应用的稳定性和安全性;在数据库层面,通过多数据库主从复制、负载均衡和备份恢复等策略实现数据的高可用和稳定性;在服务器层面,通过对服务的监控和服务器的监控、服务器间负载均衡的配置、基于API的自动化failover等策略保障服务的正常运行和高负载应对。
\section{论文组织结构}
基于Spring MVC架构WEB应用的系统优化策略的研究是按照计划逐步完成的,本人将此研究分为六个部分:

第一章:绪论。本章主要介绍WEB应用开发的现状和系统优化策略研究的意义,明确了在WEB应用开发过程中对于应用和服务器进行优化的必要性和重要性。除此之外,本章还介绍了论文研究和实验的主要内容、论文的目标、论文创新点以及本篇论文的主要结构。

第二章:WEB应用开发介绍。本章主要介绍了前端和后端的开发框架和开发流程以及数据库的搭建过程。然后介绍了Tomcat的配置过程。之后介绍了Jenkins持续集成环境的搭建和使用过程。

第三章:应用性能优化介绍。本章主要介绍了在WEB应用性能优化方面的主要方法和策略,主要包括Couchbase缓存优化,Tomcat高性能Apr配置,Docker分布式优化以及SLB负载均衡优化等方面。同时对应用的安全性和稳定性进行了分析。

第四章:数据优化。本章主要介绍了MySQL数据库在设计和开发过程的优化策略,主要包括基于数据稳定性的复制方案,基于高可用的负载均衡方案,基于数据完整性的备份和恢复方案。同时总结了出现问题时的解决方案。

第五章:服务监控与应急措施优化。本章主要介绍了服务器层面的优化策略,主要包括多服务器的心跳监听服务配置,服务性能的监控脚本实现,基于API的应急措施处置以及服务器安全性配置等方面的优化策略。同时总结了多种方案整合使用的策略。

第六章:总结与展望。本章主要总结了不同优化策略对于系统性能的影响以及在研究过程中获得的经验,并对优化策略的进一步研究做了简单的展望。